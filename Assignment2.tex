% Homework template for Inference and Information
% UPDATE: September 26, 2017 by Xiangxiang
\documentclass[a4paper]{article}
\usepackage{amsmath, amssymb, amsthm}
% amsmath: equation*, amssymb: mathbb, amsthm: proof
\usepackage{moreenum}
\usepackage{mathtools}
\usepackage{url}
\usepackage{enumitem}
\usepackage{graphicx}
\usepackage{subcaption}
\usepackage{booktabs} % toprule
\usepackage[mathcal]{eucal}
%% Definitions for Inference and Information
%% UPDATE: 22/03/2018  by zhaofeng-shu33
%% This package does not require other packages
\ProvidesPackage{dtx-style}
% semester
\newcommand{\theterm}[1]{#1}
\newcommand{\thecourseinstitute}[1]{#1}
\newcommand{\thecoursename}[1]{\textsc{#1}}

\newcommand{\courseheader}[3]{
\vspace*{-1in}
\begin{center}
\thecourseinstitute{#1}\\
\thecoursename{#3} \\
\theterm{#2}
\vspace*{0.1in}
\hrule
\end{center}
}
\def\v#1{\underline{#1}}
\newcommand{\rvx}{\mathsf{x}}    % x, r.v.
\newcommand{\rvy}{\mathsf{y}}    % y, r.v.
\newcommand{\rvz}{\mathsf{z}}    % z, r.v.
\newcommand{\rvH}{\mathsf{H}} 
\newcommand{\uy}{\underline{y}}
\newcommand{\urvx}{\underline{\mathsf{x}}}    % x, r.v. vec
\newcommand{\urvy}{\underline{\mathsf{y}}}    % y, r.v. vec
\newcommand{\urvz}{\underline{\mathsf{z}}}    % z, r.v. vec
\newcommand{\urvw}{\underline{\mathsf{w}}} 
\newcommand{\defas}{\triangleq} %\coloneqq
\newcommand{\reals}{\mathbb{R}}
\newcommand{\TT}{\mathrm{T}}    % transpose
% \newcommand{\E}[1]{\mathbb{E}\left[{#1}\right]}
% \newcommand{\Prob}[1]{\mathbb{P}\left({#1}\right)}
\DeclareMathOperator*{\argmax}{arg\,max}
\DeclareMathOperator*{\argmin}{arg\,min}
\DeclareMathOperator*{\argsup}{arg\,sup}
\DeclareMathOperator*{\arginf}{arg\,inf}
\DeclareMathOperator{\diag}{diag}
\DeclareMathOperator{\Var}{Var}
\DeclareMathOperator{\Cov}{Cov}
\DeclareMathOperator{\MSE}{MSE}
\DeclareMathOperator{\In}{\mathbb{I}}
\DeclareMathOperator{\E}{\mathbb{E}}
\DeclareMathOperator{\Prob}{\mathbb{P}}
\newcommand\independent{\protect\mathpalette{\protect\independenT}{\perp}}
\def\independenT#1#2{\mathrel{\rlap{$#1#2$}\mkern2mu{#1#2}}}
\endinput

\begin{document}
\courseheader

\newcounter{hwcnt}
\setcounter{hwcnt}{2} % set to the times of Homework

\begin{center}
  \underline{\bf Homework \thehwcnt} \\
\end{center}
\begin{flushleft}
  赵丰\hfill
  \today
\end{flushleft}
\hrule

\vspace{2em}

\flushleft
\rule{\textwidth}{1pt}
\begin{itemize}
\item {\bf Acknowledgments: \/} 
  This coursework referes to wikipedia: \small{\url{https://en.wikipedia.org}}.

\item {\bf Collaborators: \/}
  I finish this coursework by myself.
\end{itemize}
\rule{\textwidth}{1pt}

\vspace{2em}

I use \texttt{enumerate} to generate answers for each question:

\begin{enumerate}[label=\thehwcnt.\arabic*.]
  \setlength{\itemsep}{3\parskip}

  \item 
    \begin{enumerate}[label=(\alph*)]
    \item 
    \begin{enumerate}[label=\roman*.]
      \item 
         对向量值随机变量证明全(协)方差公式。
         用到$\E[\urvx]=\E[\E[\urvx|\rvy]]$以及
         $\E[\urvx\urvx^\TT]=\E[\E[\urvx|\rvy]]$
        \begin{align*}
        \Cov(\urvx) = & \E[(\urvx-\E[\urvx])(\urvx-\E[\urvx])^\TT] \\    
                    = & \E[\urvx\urvx^\TT]-\E[\urvx]\E[\urvx]^\TT\\
                    = & \E[\E[\urvx\urvx^\TT|\rvy]]-\E[\E[\urvx|\rvy]]\E[\E[\urvx|\rvy]]^\TT\\
                    = & \E[\Cov[\urvx|\rvy]+\E[\urvx|\rvy]\E[\urvx|\rvy]^\TT]-\E[\E[\urvx|\rvy]]\E[\E[\urvx|\rvy]]^\TT\\
                    = & \E[\Cov[\urvx|\rvy]]+\Cov[\E[\urvx|\rvy]]\\       
        \end{align*}
        所以
        \begin{equation}
        \E[\rvx|\rvy]=\E[\E[\rvx|\rvy \rvz]|\rvy]
        \end{equation}

      \item 
        若$\exists \v{c}\in \mathbb{R}^k,\v{c}\neq \v{0},d\in \mathbb{R}$使得$\v{c}^\TT \urvx =d$.
        则 $\v{c}^\TT \E[\urvx] =d$且有$\v{c}^\TT \urvx \urvx^\TT \v{c} =d^2$,取期望得
        $\v{c}^\TT \E[\urvx \urvx^\TT] \v{c} =d^2$
        \begin{align*}
        \v{c}^\TT \Cov(\urvx)\v{c} = & \v{c}^\TT\E[\urvx\urvx^\TT]\v{c}-\v{c}^\TT\E[\urvx]\E[\urvx]^\TT\v{c}\\
        = & \v{c}^\TT\E[\urvx\urvx^\TT]\v{c}-\v{c}^\TT\E[\urvx]\E[\urvx]^\TT\v{c}\\
        = & 0
        \end{align*}
        因为$\Cov(\urvx)$是半正定矩阵,从上式知其有一个特征值为零。从而其行列式为0。
        
        反之,若$\det[\Cov(\urvx)]=0$,则$\exists \v{c}\in \mathbb{R}^k,\v{c}\neq \v{0}$,使得
        $\v{c}^\TT \Cov(\urvx)\v{c}=0$,设$\rvy=\v{c}^\TT(\urvx-\E[\urvx])$
        即 $\E[\rvy^2]=0$,从而$\rvy=0$,取$d=\v{c}^\TT\E[\urvx]$,则有$\v{c}^\TT\urvx=d$
      
    \end{enumerate}
    \item 
      \begin{enumerate}[label=\roman*.]
        \item
            \begin{align*}
            \MSE(\v{\alpha}) = & \sum_{i=1}^k [(1-\v{\alpha}_i)^2+\sum_{\substack{j=1\\j\neq i}}^k \v{\alpha}_j^2]P_y(i) \\
            = & 1 \sum_{i=1}^k (- 2\v{\alpha}_i +\sum_{j=1}^k \v{\alpha}_j^2) P_y(i)\\
            = & 1- 2 \sum \v{\alpha}_i P_y(i) + \sum_{j=1}^k (\sum_{i=1}^k P_y(i))\v{\alpha}_j^2\\
            = & 1- 2 \sum_{i=1}^k (\v{\alpha}_i^2-2\v{\alpha}_i P_y(i))
            \end{align*}
        根据二次函数极值的性质,当$\v{\alpha}_i=P_y(i)$时上式最小。即题中所给$\v{P}_y(\cdot)$是MMSE估计量。
        \item
            类似上次作业,只需对给定的$\urvx=\v{x}_0$极小化条件概率下的误差期望:
            \begin{equation}
            \E_{\urvy|\urvx=\v{x}_0}[||\urvy-\v{f}(\urvx=\v{x}_0)||_2^2 |]            
            \end{equation}
            此时$\v{f}(\urvx=\v{x}_0)$视为常数,由前一小问结论,当
            $\v{f}(\urvx=\v{x}_0)=\v{P}_{\rvy}(\cdot|\urvx=\v{x}_0)$时上式最小,所以MMSE估计器是
            $\v{P}_{\rvy}(\cdot|\urvx)$
                   
      \end{enumerate}      
    \end{enumerate}
  \item 
  \begin{enumerate}[label=(\alph*)]
  \item
    $\Var(\v{v}^\TT \urvx)=\v{v}^\TT\Cov(\v{x})\v{v}$
    设$U^\TT=(\v{u}_1,\v{u}_2,\dots,\v{u}_d)$,则$\v{u}_i$是$\sigma_i$对应的特征向量。
    由Rayleigh 商的性质可说明$\v{v}_i=\v{u}_i$。设$\v{v}=\sum_{i=1}^d \alpha_i \v{u}_i$
    \begin{align}
    \Var(\v{v}^\TT \urvx)=& \frac{\v{v}^\TT\Cov(\v{x})\v{v}}{||\v{v}||^2}\nonumber\\
    = & \frac{(\sum_{i=1}^d \alpha_i \v{u}_i)^\TT(\sum_{i=1}^d \sigma_i\alpha_i \v{u}_i)}{(\sum_{i=1}^d \alpha_i \v{u}_i)^\TT(\sum_{i=1}^d \alpha_i \v{u}_i)}\nonumber\\
    = & \frac{\sum_{i=1}^d \alpha^2_i \sigma_i}{\sum_{i=1}^d \alpha^2_i}\label{eq:RayleighQuotient}
    \end{align}
    注意到上式是$\sigma_1,\dots,\sigma_d$的凸组合,在无正交性约束的情况下取$\alpha_i=\delta_{1i}$使得$\sigma_1$前的系数最大即得到上式的最大值,
    此时
    \begin{equation}
    \v{v}_1=\argmax_{\v{v}:||\v{v}||=1} \Var(\v{v}^\TT \urvx)=\v{u}_1
    \end{equation}
    若加入正交性约束$\v{v}\perp\v{u}_1$,则$\alpha_1=0$,则在\eqref{eq:RayleighQuotient}式中相当于$\sigma_2,\dots,\sigma_d$的凸组合。
    因此取$\alpha_i=\delta_{2i}$使得$\sigma_2$前的系数最大得到有正交性约束$\v{v}\perp\v{u}_1$情形下的最大值。即推出$\v{v}_2=\v{u}_2$
    如此递推下去即可证明$\v{v}_i=\v{u}_i$。
  \item
    注意到$\Cov(\rvy_i,\rvy_j)=\v{v}_i^T \Cov(\urvx)\v{v}_j=\sigma_j \delta_{ij}$
    因此$\Cov(\urvy)=\Sigma$。
  \item
    \begin{enumerate}[label=\roman*.]
        \item $\urvy$也是多维正态分布,$\E[\urvy]=0$,其联合概率
        密度函数可以写为:
        \begin{equation}
            p_{\urvy}(\v{y})=\prod_{i=1}^d \frac{1}{\sqrt{2\pi \sigma_i^2}}\exp(-\frac{y_i^2}{2\sigma_i^2})
        \end{equation}
        \item 由于$\urvy$的各分量不相关,所以彼此独立。
    \end{enumerate}
  \end{enumerate}  
  \item  
  \begin{enumerate}[label=(\alph*)]
    \item 
    \begin{equation}K_{\rvx}=
        \begin{bmatrix}
         \sigma^2 & \rho \sigma \\
         \rho \sigma & \sigma^2
        \end{bmatrix}
    \end{equation}
    \begin{equation}K_{\rvy}=A K_{\rvx} A^\TT
        \begin{bmatrix}
         \sigma^2 & \rho_{\rvx} \sigma^2 \\
         \rho_{\rvx} \sigma^2 & \sigma^2
        \end{bmatrix}=\sigma^2\diag\{1-\rho_{\rvx}^2,1\}
    \end{equation}
    \item 由$K_{\rvy}$是对角阵以及$\urvy$服从二元正态分布可知$\rvy_1 \independent \rvy_2$
    所以$\rvy_1 \independent g(\rvy_2)$,即$\Cov(\rvy_1,g(\rvy_2))=0$,从而结论得证。
    \item 
    为证明 $\E[\rvy_1^2]\leq \E[(\rvy_1 + \rho_x \rvy_2-g(\rvy_2))^2]$
    因为$\rvy_1$与$\rho_x \rvy_2-g(\rvy_2)$相互独立,故上式右端为$\E[\rvy_1^2]+\E[(\rho_x \rvy_2-g(\rvy_2))^2]$,
    所以不等式成立。
    \end{enumerate}
  \item 上机作业  
  \item Thanks to 陆石, who gives me this template.
  

  \end{enumerate}

\end{document}
\begin{equation}
\end{equation}

%%% Local Variables:
%%% mode: late\rvx
%%% TeX-master: t
%%% End:
