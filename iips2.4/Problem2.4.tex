% Homework template for Inference and Information
% UPDATE: September 26, 2017 by Xiangxiang
\documentclass[a4paper]{article}
\usepackage{amsmath, amssymb, amsthm}
% amsmath: equation*, amssymb: mathbb, amsthm: proof
\usepackage{moreenum}
\usepackage{mathtools}
\usepackage{url}
\usepackage{enumitem}
\usepackage{graphicx}
\usepackage{subcaption}
\usepackage{booktabs} % toprule
\usepackage[mathcal]{eucal}
%% Definitions for Inference and Information
%% UPDATE: 22/03/2018  by zhaofeng-shu33
%% This package does not require other packages
\ProvidesPackage{dtx-style}
% semester
\newcommand{\theterm}[1]{#1}
\newcommand{\thecourseinstitute}[1]{#1}
\newcommand{\thecoursename}[1]{\textsc{#1}}

\newcommand{\courseheader}[3]{
\vspace*{-1in}
\begin{center}
\thecourseinstitute{#1}\\
\thecoursename{#3} \\
\theterm{#2}
\vspace*{0.1in}
\hrule
\end{center}
}
\def\v#1{\underline{#1}}
\newcommand{\rvx}{\mathsf{x}}    % x, r.v.
\newcommand{\rvy}{\mathsf{y}}    % y, r.v.
\newcommand{\rvz}{\mathsf{z}}    % z, r.v.
\newcommand{\rvH}{\mathsf{H}} 
\newcommand{\uy}{\underline{y}}
\newcommand{\urvx}{\underline{\mathsf{x}}}    % x, r.v. vec
\newcommand{\urvy}{\underline{\mathsf{y}}}    % y, r.v. vec
\newcommand{\urvz}{\underline{\mathsf{z}}}    % z, r.v. vec
\newcommand{\urvw}{\underline{\mathsf{w}}} 
\newcommand{\defas}{\triangleq} %\coloneqq
\newcommand{\reals}{\mathbb{R}}
\newcommand{\TT}{\mathrm{T}}    % transpose
% \newcommand{\E}[1]{\mathbb{E}\left[{#1}\right]}
% \newcommand{\Prob}[1]{\mathbb{P}\left({#1}\right)}
\DeclareMathOperator*{\argmax}{arg\,max}
\DeclareMathOperator*{\argmin}{arg\,min}
\DeclareMathOperator*{\argsup}{arg\,sup}
\DeclareMathOperator*{\arginf}{arg\,inf}
\DeclareMathOperator{\diag}{diag}
\DeclareMathOperator{\Var}{Var}
\DeclareMathOperator{\Cov}{Cov}
\DeclareMathOperator{\MSE}{MSE}
\DeclareMathOperator{\In}{\mathbb{I}}
\DeclareMathOperator{\E}{\mathbb{E}}
\DeclareMathOperator{\Prob}{\mathbb{P}}
\newcommand\independent{\protect\mathpalette{\protect\independenT}{\perp}}
\def\independenT#1#2{\mathrel{\rlap{$#1#2$}\mkern2mu{#1#2}}}
\endinput

\begin{document}
\courseheader

\newcounter{hwcnt}
\setcounter{hwcnt}{2} % set to the times of Homework

\begin{center}
  \underline{\bf Homework \thehwcnt} \\
\end{center}
\begin{flushleft}
  赵丰\hfill
  \today
\end{flushleft}
\hrule

\vspace{2em}

\flushleft
\rule{\textwidth}{1pt}
\begin{itemize}
\item {\bf Acknowledgments: \/} 
  This coursework referes to wikipedia: \small{\url{https://en.wikipedia.org}}.

\item {\bf Collaborators: \/}
  I finish this coursework by myself.
\end{itemize}
\rule{\textwidth}{1pt}

\vspace{2em}

I use \texttt{enumerate} to generate answers for each question:

\begin{enumerate}[label=\thehwcnt.\arabic*.]
  \setlength{\itemsep}{3\parskip}
  \setcounter{enumi}{3}
  \item 
    \begin{enumerate}[label=(\alph*)]
    \item 见相关代码
    \item 
        \begin{enumerate}[label=\roman*.]
        \item $\urvw$与$\urvx$分布相同。
        \item 见相关代码
        \item 计算结果表明$\E[\urvw^\TT\urvz]\approx 0$,即$\urvw$与$\urvz$是彼此独立的。
        $\E[\urvx^\TT\urvz]=\E[\urvz^\TT C \urvz]$明显大于零。
        \end{enumerate}   
    \item 见相关代码
    \item 
        \begin{enumerate}[label=\roman*.]
        \item $\rvx_1$在$\rvx_2=0$的条件下是高斯分布,由$\E[\rvx_1|\rvx_2]=\rho \rvx_2$知其均值为0,
        由联合高斯条件分布的方差公式知方差为$(1-\rho^2)\sigma^2$
        所以条件 PDF 为
        \begin{equation}
            p_{\rvx_1|\rvx_2}(x|0)=\frac{1}{\sqrt{2\pi(1-\rho^2)\sigma^2}}\exp(-\frac{x^2}{2(1-\rho^2)\sigma^2})
        \end{equation}
        \item 见相关代码
        \item 见相关代码,仿真结果与理论结果一致。
        \end{enumerate}       
    \item 见相关代码
    \end{enumerate}   
  \item Thanks to 陆石, who gives me this template.
  

\end{enumerate}

\end{document}
\begin{equation}
\end{equation}

%%% Local Variables:
%%% mode: late\rvx
%%% TeX-master: t
%%% End:
