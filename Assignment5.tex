% Homework template for Inference and Information
% UPDATE: September 26, 2017 by Xiangxiang
\documentclass[a4paper]{article}
\usepackage{amsmath, amssymb, amsthm}
% amsmath: equation*, amssymb: mathbb, amsthm: proof
\usepackage{ctex}
\usepackage{moreenum}
\usepackage{mathtools}
\usepackage{url}
\usepackage{bm}
\usepackage{enumitem}
\usepackage{graphicx}
\usepackage{subcaption}
\usepackage{booktabs} % toprule
\usepackage[mathcal]{eucal}
\usepackage{iidef}
\begin{document}
% institute-time-course_name
\courseheader{清华-伯克利深圳学院}{2017年秋季学期}{信息推断}

\newcounter{hwcnt}
\setcounter{hwcnt}{5} % set to the times of Homework

\begin{center}
  \underline{\bf Homework \thehwcnt} \\
\end{center}
\begin{flushleft}
  赵丰\hfill
  \today
\end{flushleft}
\hrule

\vspace{2em}

\flushleft
\rule{\textwidth}{1pt}
\begin{itemize}
\item {\bf Acknowledgments: \/} 
  This coursework referes to wikipedia: \small{\url{https://en.wikipedia.org}}.

\item {\bf Collaborators: \/}
  I finish this coursework by myself.
\end{itemize}
\rule{\textwidth}{1pt}

\vspace{2em}

\begin{enumerate}[label=\thehwcnt.\arabic*.]
  \setlength{\itemsep}{3\parskip}

  \item 
  \[
  D(p_{\rvy}(\cdot;x)||p_{\rvy}(\cdot;x+\delta))=
  \sum_{y} p_{\rvy}(y;x) \ln \frac{p_{\rvy}(y;x)}{p_{\rvy}(y;x+\delta)}
  \]
  设 $F(x)=\ln (p_{\rvy}(\cdot;x))$
  则 
  \[
  F(x+\delta)=F(x)+F'(x)\delta+\frac{1}{2}F''(x)\delta^2
  \]
  因此
  \[
  D(p_{\rvy}(\cdot;x)||p_{\rvy}(\cdot;x+\delta))=
  -\sum_{y} p_{\rvy}(y;x) (F'(x)\delta+\frac{1}{2}F''(x)\delta^2)+O(\delta^3)
  \]
  注意到
  $$
  F'(x)=\frac{1}{p_{\rvy}(y;x)} \frac{\partial}{\partial x} p_{\rvy}(y;x)
  $$
  因此
  \begin{align}
  \sum_{y} p_{\rvy}(y;x) F'(x)=&\sum_{y} \frac{\partial}{\partial x} p_{\rvy}(y;x)\\
  = & \frac{\partial}{\partial x} \sum_{y}  p_{\rvy}(y;x) \\
  = & 0
  \end{align}
  $\Rightarrow$
  \[
   \frac{D(p_{\rvy}(\cdot;x)||p_{\rvy}(\cdot;x+\delta))}{\delta^2}=
   -\frac{1}{2}\sum_{y} p_{\rvy}(y;x) F''(x)+O(\delta)
  \]
  $\Rightarrow$
  \[
   \lim_{\delta \to 0} \frac{D(p_{\rvy}(\cdot;x)||p_{\rvy}(\cdot;x+\delta))}{\delta^2}=
   \frac{1}{2}J_{\rvy}(x)  
  \]  
  
  \item 
  \begin{enumerate}[label=(\alph*)]
  \item 
   \[
   J_{\urvy}=\begin{cases}
   2 & x>0\\
   \frac{3}{2} & x<0
   \end{cases}
   \]
   评注: Fisher Information 具有可加性。
  
  \item
  当$x>0$时,可以求出$\Var[\hat{x_1}(\rvy)]=\frac{1}{2}$,达到 CRB下界;
  当$x<0$时,可以求出$\Var[\hat{x_2}(\rvy)]=\frac{2}{3}$,达到 CRB下界;
  因此如果$x\in \mathbb{R}/\{0\}$,存在MVU 估计量,那么它会达到CRB下界,从而同时具有
  $\hat{x_1}(\rvy)$和$\hat{x_1}(\rvy)$的形式,矛盾。

  注:有效估计量具有形式
  \begin{align*}
   \hat{x}(y) = & x + \frac{1}{J_{\rvy}(x)} \frac{\partial}{\partial x}\ln p_{\rvy}(\rvy;x) \\
               = & x + \frac{1}{\frac{1}{\Var(\mathsf{w}_1)}+\frac{1}{\Var(\mathsf{w}_2)}}\left[
    \frac{\rvy_1-x}{\Var(\mathsf{w}_1)} + \frac{\rvy_2-x}{\Var(\mathsf{w}_2)} \right] \\
               = & \begin{cases}
                    \frac{1}{2}\rvy_1 +\frac{1}{2}\rvy_2 ,& x>0 \\
                    \frac{2}{3}\rvy_1 +\frac{1}{3}\rvy_2 ,& x<0
                    \end{cases}
  \end{align*}
  
  \end{enumerate} 
  \item
  \begin{enumerate}[label=(\alph*)]
  \item 
  设$P(H_1)=q$,似然比检验准则的阈值$\lambda_B=\frac{1-q}{q}$,
$$
\frac{p(\rvy|H_1)}{p(\rvy|H_0)}\mathop{\gtreqless}_{\hat{H}(\rvy)=H_0}^{\hat{H}(\rvy)=H_1} \lambda_B
\Rightarrow y \mathop{\gtreqless}_{\hat{H}(\rvy)=H_0}^{\hat{H}(\rvy)=H_1} \frac{\sigma^2 \ln \lambda_B}{s_1-s_0}
+\frac{s_0+s_1}{2}=:y_0
$$
\begin{align*}
P_F = & \int_{y_0}^{\infty} p(y|H_0) dy = \Phi(-\frac{y_0-s_0}{\sigma}) \\
P_M = & \int_{-\infty}^{y_0} p(y|H_1) dy = \Phi(\frac{y_0-s_1}{\sigma})
\end{align*}
其中$\Phi$是标准正态分布的分布函数,$\Phi(x)+Q(x)=1$。

由极大极小化方程和对称代价得$P_F=P_M \Rightarrow y_0 =\frac{s_0+s_1}{2} \Rightarrow \lambda_B=0\Rightarrow \xi=\frac{1}{2}$
  判决准则是:
  $$
  y \mathop{\gtreqless}_{\hat{H}(y)=H_0}^{\hat{H}(y)=H_1} \frac{s_0+s_1}{2} 
  $$
  
  \item
  $p$是实际的先验,$q$是用于设计 Bayes 判决准则的先验,则代价函数为
  \begin{align}
  \varphi_{\mathrm{B}}(q,p) = & (1-p)P_{\mathrm{F}}(q)+p P_{\mathrm{M}}(q)) \\
  = & (1-p)\Phi(-[\frac{\sigma \ln \lambda_B(q)}{s_1-s_0}+\frac{s_1-s_0}{2\sigma}])+p \Phi(\frac{\sigma \ln \lambda_B(q)}{s_1-s_0}-\frac{s_1-s_0}{2\sigma})
  \end{align}

  将$q=\frac{1}{2}$和$q=p$分别代入Bayes风险函数$\varphi_B(q,p)$ 

  并代入数字,得$\varphi_{\mathrm{B}}(\frac{1}{2},p)=0.16,\varphi_{\mathrm{B}}(p,p)=0.08$
  其中前者是极大极小准则的期望代价,后者是已知先验$p$时Bayes准则的期望代价。
  \end{enumerate}
  \item 
   \begin{enumerate}[label=(\alph*)]
   \item $p_{\rvz}(z;x)$ 也属于指数分布族,分布函数为  $ \exp[\lambda(x)t(z-a)-\alpha(x)+\beta(z-a)] $。
   \item 
    \begin{align*}
    p_{\rvz}(z;x) = & p_{\rvy}* p_{\rvy}(z) \\
                  = & \int_{\mathbb{R}} p_{\rvy}(y;x)p_{\rvy}(z-y;x)dy \\
                  = & \exp(xz-\alpha'(x)+\beta'(z))
    \end{align*}
    其中$\alpha'(x)=2\alpha(x),\beta'(z)=\ln \int_{\mathbb{R}} \exp[\beta(y)+\beta(z-y)]dy$
    \end{enumerate}  

  \item Thanks to 陆石, who gives me this template.
  

  \end{enumerate}

\end{document}
\begin{equation}
\end{equation}

%%% Local Variables:
%%% mode: late\rvx
%%% TeX-master: t
%%% End:
