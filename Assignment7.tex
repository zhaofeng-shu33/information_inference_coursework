% Homework template for Inference and Information
% UPDATE: September 26, 2017 by Xiangxiang
\documentclass[a4paper]{article}
\usepackage{amsmath, amssymb, amsthm}
% amsmath: equation*, amssymb: mathbb, amsthm: proof
\usepackage{moreenum}
\usepackage{mathtools}
\usepackage{url}
\usepackage{bm}
\usepackage{enumitem}
\usepackage{graphicx}
\usepackage{subcaption}
\usepackage{booktabs} % toprule
\usepackage[mathcal]{eucal}
\usepackage{amsmath}
%% Definitions for Inference and Information
%% UPDATE: 22/03/2018  by zhaofeng-shu33
%% This package does not require other packages
\ProvidesPackage{dtx-style}
% semester
\newcommand{\theterm}[1]{#1}
\newcommand{\thecourseinstitute}[1]{#1}
\newcommand{\thecoursename}[1]{\textsc{#1}}

\newcommand{\courseheader}[3]{
\vspace*{-1in}
\begin{center}
\thecourseinstitute{#1}\\
\thecoursename{#3} \\
\theterm{#2}
\vspace*{0.1in}
\hrule
\end{center}
}
\def\v#1{\underline{#1}}
\newcommand{\rvx}{\mathsf{x}}    % x, r.v.
\newcommand{\rvy}{\mathsf{y}}    % y, r.v.
\newcommand{\rvz}{\mathsf{z}}    % z, r.v.
\newcommand{\rvH}{\mathsf{H}} 
\newcommand{\uy}{\underline{y}}
\newcommand{\urvx}{\underline{\mathsf{x}}}    % x, r.v. vec
\newcommand{\urvy}{\underline{\mathsf{y}}}    % y, r.v. vec
\newcommand{\urvz}{\underline{\mathsf{z}}}    % z, r.v. vec
\newcommand{\urvw}{\underline{\mathsf{w}}} 
\newcommand{\defas}{\triangleq} %\coloneqq
\newcommand{\reals}{\mathbb{R}}
\newcommand{\TT}{\mathrm{T}}    % transpose
% \newcommand{\E}[1]{\mathbb{E}\left[{#1}\right]}
% \newcommand{\Prob}[1]{\mathbb{P}\left({#1}\right)}
\DeclareMathOperator*{\argmax}{arg\,max}
\DeclareMathOperator*{\argmin}{arg\,min}
\DeclareMathOperator*{\argsup}{arg\,sup}
\DeclareMathOperator*{\arginf}{arg\,inf}
\DeclareMathOperator{\diag}{diag}
\DeclareMathOperator{\Var}{Var}
\DeclareMathOperator{\Cov}{Cov}
\DeclareMathOperator{\MSE}{MSE}
\DeclareMathOperator{\In}{\mathbb{I}}
\DeclareMathOperator{\E}{\mathbb{E}}
\DeclareMathOperator{\Prob}{\mathbb{P}}
\newcommand\independent{\protect\mathpalette{\protect\independenT}{\perp}}
\def\independenT#1#2{\mathrel{\rlap{$#1#2$}\mkern2mu{#1#2}}}
\endinput

\begin{document}
\courseheader
\newcounter{hwcnt}
\setcounter{hwcnt}{7} % set to the times of Homework

\begin{center}
  \underline{\bf Homework \thehwcnt} \\
\end{center}
\begin{flushleft}
  赵丰\hfill
  \today
\end{flushleft}
\hrule

\vspace{2em}

\flushleft
\rule{\textwidth}{1pt}
\begin{itemize}
\item {\bf Acknowledgments: \/} 
  This coursework referes to wikipedia: \small{\url{https://en.wikipedia.org}}.

\item {\bf Collaborators: \/}
  I finish this coursework by myself.
\end{itemize}
\rule{\textwidth}{1pt}

\vspace{2em}

I use \texttt{enumerate} to generate answers for each question:

\begin{enumerate}[label=\thehwcnt.\arabic*.]
  \setlength{\itemsep}{3\parskip}

  \item 
    \begin{enumerate}[label=(\alph*)]
  \item 
    $D(q||p) \geq q(M)\log\frac{q(M)}{p(M)}=\infty$
    所以$D(q||p)=\infty$
  \item 因为 $\displaystyle\lim_{x\to 0+} x\log(x)=0$,
   $D(p||q) =\displaystyle\sum_{y=0}^{M-1} p(y)\log\frac{p(y)}{q(y)}<\infty$
  \item 运用 Lagrange 乘子法可以求出 $p(k)=t q(k),k=0,\dots,M-1$,通过归一化条件求出
  $t=\frac{1}{Q(M-1)}$, 另外由函数的凸性可得局部最小为全局最小,从而
  \[
   p^*(k)=\frac{q(k)}{Q(M-1)},k=0,\dots,M-1 \textrm{ and } D(p^* || q )=\log\frac{1}{Q(M-1)}
  \]
   另解:
   \begin{align*}
   D(p||q) = & \sum_{y=0}^{M-1} p(y) \log \frac{p(y)}{\frac{q(y)}{Q(M-1)}Q(M-1)}\\
           = & \sum_{y=0}^{M-1} D\left(p|| \frac{q(y)}{Q(M-1)} 1_{y<M}\right)+\log \frac{1}{Q(M-1)}
   \end{align*}
  \item 对
  \[
   f(p)=\sum_{m=0}^{\infty} p_m \log \frac{p_m}{q_m} -\lambda (
   \sum_{m=0}^{\infty} p_m -1) -\mu (\sum_{m=M}^{\infty} p_m -\epsilon)
  \]
  运用 Lagrange 乘子法求出
  \begin{align*}
    p^*_k =  t q_k,& m \leq M-1 \\
    p^*_k =  t' q_k,& m \geq M
   \end{align*}
   由归一化条件和$\mathcal{P}_{\epsilon}$的性质可得
  \begin{align*}
    t = &  \frac{1-\epsilon}{Q(M-1)} \\
    t'= &  \frac{\epsilon}{1-Q(M-1)}
   \end{align*}
   从而得到 $D(p^*_{\epsilon}||q)=(1-\epsilon)\log t+\epsilon\log t'$
   并且有:
   \[
    \lim_{\epsilon\to 0+}D(p^*_{\epsilon}||q)=D(p^*||q)
   \]
   \item $t(y)=1(y\geq M),c=\epsilon$
   \item $p_{\epsilon}^*(y) = q(y)\exp(xt(y)-\alpha(x))\Rightarrow e^x = \frac{t'}{t}$
   所以 $$ x=\log\frac{\epsilon Q(M-1)}{(1-\epsilon)(1-Q(M-1))}$$
   \end{enumerate}
   \item
   \begin{enumerate}[label=(\alph*)]
   \item 我们证明$D(p_{\rvx,\rvy}||q_{\rvx}q_{\rvy})\geq D(p_{\rvx,\rvy}||p_{\rvx}p_{\rvy})$
   \begin{align*}
   D(p_{\rvx,\rvy}||q_{\rvx}q_{\rvy}) - D(p_{\rvx,\rvy}||p_{\rvx}p_{\rvy}) = & \sum_{x,y}p_{\rvx,\rvy}(x,y)\log\frac{p_{\rvx}(x)p_{\rvy}(y)}{q_{\rvx}(x)q_{\rvy}(y)}\\
   = & \sum_{x,y} p_{\rvx,\rvy}(x,y) \log\frac{p_{\rvx}(x)}{q_{\rvx}(x)} + \sum_{\rvx,\rvy} p_{\rvx,\rvy}(x,y) \log\frac{p_{\rvy}(y)}{q_{\rvy}(y)}\\
   = & \sum_{x} p_{\rvx}(x) \log\frac{p_{\rvx}(x)}{q_{\rvx}(x)} + \sum_{y} p_{\rvy}(y) \log\frac{p_{\rvy}(y)}{q_{\rvy}(y)} \\
   = & D(p_{\rvx} || q_{\rvx}) + D(p_{\rvy} || q_{\rvy})\geq 0
   \end{align*}
   当 $q_x=p_x,q_y=p_y$时取等号,因此 $\min D(p_{\rvx,\rvy}||q_{\rvx}q_{\rvy})=I(\rvx,\rvy)$
   \item 注意到 $p=0 \Rightarrow q=0,$ 否则$D(q||p_{\rvx,\rvy})=\infty$。
   所以$q_{\rvx}(4)q_{\rvy}(k)=0,k=1,2,3$,若$q_{\rvx}(4)\neq 0$,则$q_{\rvy}=(0,0,0,1),q_{\rvx}=(0,0,0,1)\Rightarrow D(q||p_{\rvx,\rvy})=\log 4$;
   若$q_{\rvx}(4)= 0\Rightarrow q_{\rvy}(4)=0$,因为$q_{\rvx}(3)q_{\rvy}(k)=0,k=1,2$,若$q_{\rvx}(3)\neq 0$,则$q_{\rvy}=(0,0,1,0),q_{\rvx}=(0,0,1,0)\Rightarrow D(q||p_{\rvx,\rvy})=\log 4$;若$q_{\rvx}(3)=0\Rightarrow q_{\rvy}(3)=0$。
   为记号简便,设$q_{\rvx}(1)=a,q_{\rvy}(1)=b$,则只需极小化下面的函数
\[
f(a,b)=ab\log(8ab)+(1-a)(1-b)\log(8(1-a)(1-b))+a(1-b)\log(8a(1-b))+b(1-a)\log(8b(1-a))
\]
化简得:
\[
f(a,b)=  \log 8 + a\log a +(1-a)\log(1-a) + b\log b +(1-b)\log(1-b)
\]
当$a=b=\frac{1}{2}$时,$f(a,b)$取得最小值$\log 2$
而(a)中的解使得$D(q||p_{\rvx,\rvy})=\infty$。
   \end{enumerate}

   
  \end{enumerate}
\end{document}


%%% Local Variables:
%%% mode: late\rvx
%%% TeX-master: t
%%% End:
