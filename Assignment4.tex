% Homework template for Inference and Information
% UPDATE: September 26, 2017 by Xiangxiang
\documentclass[a4paper]{article}
\usepackage{amsmath, amssymb, amsthm}
% amsmath: equation*, amssymb: mathbb, amsthm: proof
\usepackage{moreenum}
\usepackage{mathtools}
\usepackage{url}
\usepackage{bm}
\usepackage{enumitem}
\usepackage{graphicx}
\usepackage{subcaption}
\usepackage{booktabs} % toprule
\usepackage[mathcal]{eucal}
%% Definitions for Inference and Information
%% UPDATE: 22/03/2018  by zhaofeng-shu33
%% This package does not require other packages
\ProvidesPackage{dtx-style}
% semester
\newcommand{\theterm}[1]{#1}
\newcommand{\thecourseinstitute}[1]{#1}
\newcommand{\thecoursename}[1]{\textsc{#1}}

\newcommand{\courseheader}[3]{
\vspace*{-1in}
\begin{center}
\thecourseinstitute{#1}\\
\thecoursename{#3} \\
\theterm{#2}
\vspace*{0.1in}
\hrule
\end{center}
}
\def\v#1{\underline{#1}}
\newcommand{\rvx}{\mathsf{x}}    % x, r.v.
\newcommand{\rvy}{\mathsf{y}}    % y, r.v.
\newcommand{\rvz}{\mathsf{z}}    % z, r.v.
\newcommand{\rvH}{\mathsf{H}} 
\newcommand{\uy}{\underline{y}}
\newcommand{\urvx}{\underline{\mathsf{x}}}    % x, r.v. vec
\newcommand{\urvy}{\underline{\mathsf{y}}}    % y, r.v. vec
\newcommand{\urvz}{\underline{\mathsf{z}}}    % z, r.v. vec
\newcommand{\urvw}{\underline{\mathsf{w}}} 
\newcommand{\defas}{\triangleq} %\coloneqq
\newcommand{\reals}{\mathbb{R}}
\newcommand{\TT}{\mathrm{T}}    % transpose
% \newcommand{\E}[1]{\mathbb{E}\left[{#1}\right]}
% \newcommand{\Prob}[1]{\mathbb{P}\left({#1}\right)}
\DeclareMathOperator*{\argmax}{arg\,max}
\DeclareMathOperator*{\argmin}{arg\,min}
\DeclareMathOperator*{\argsup}{arg\,sup}
\DeclareMathOperator*{\arginf}{arg\,inf}
\DeclareMathOperator{\diag}{diag}
\DeclareMathOperator{\Var}{Var}
\DeclareMathOperator{\Cov}{Cov}
\DeclareMathOperator{\MSE}{MSE}
\DeclareMathOperator{\In}{\mathbb{I}}
\DeclareMathOperator{\E}{\mathbb{E}}
\DeclareMathOperator{\Prob}{\mathbb{P}}
\newcommand\independent{\protect\mathpalette{\protect\independenT}{\perp}}
\def\independenT#1#2{\mathrel{\rlap{$#1#2$}\mkern2mu{#1#2}}}
\endinput

\begin{document}
\courseheader

\newcounter{hwcnt}
\setcounter{hwcnt}{4} % set to the times of Homework

\begin{center}
  \underline{\bf Homework \thehwcnt} \\
\end{center}
\begin{flushleft}
  赵丰\hfill
  \today
\end{flushleft}
\hrule

\vspace{2em}

\flushleft
\rule{\textwidth}{1pt}
\begin{itemize}
\item {\bf Acknowledgments: \/} 
  This coursework referes to wikipedia: \small{\url{https://en.wikipedia.org}}.

\item {\bf Collaborators: \/}
  I finish this coursework by myself.
\end{itemize}
\rule{\textwidth}{1pt}

\vspace{2em}

I use \texttt{enumerate} to generate answers for each question:

\begin{enumerate}[label=\thehwcnt.\arabic*.]
  \setlength{\itemsep}{3\parskip}

  \item 
  \begin{align*}
   \E[\textsf{m}_n]=&\E[\frac{1}{n} \sum_{i=1}^n \rvx_i]\\
   =& \frac{1}{n}\sum_{i=1}^n \E[\rvx_i]\\
   =& \mu
  \end{align*}
  因此$\textsf{m}_n$是$\mu$的无偏估计量。
  
  由$\rvx_i$与$rvx_j(i\neq j)$相互独立$\Rightarrow$
  \[
  E[(\rvx_i-\mu)(\rvx_j-\mu)]=E[\rvx_i-\mu][\rvx_j-\mu]=0
  \]
  \begin{align*}
   \E[(\textsf{m}_n-\mu)^2]=&\E\left[\left(\frac{1}{n} \sum_{i=1}^n (\rvx_i-\mu)\right)^2\right]\\
   =& \frac{1}{n^2}\sum_{i=1}^n \E[(\rvx_i-\mu)^2]+\frac{1}{n^2}\sum_{i,j=1,i\neq j}^n E[(\rvx_i-\mu)(\rvx_j-\mu)]\\
   =& \frac{\sigma^2}{n}
  \end{align*}  
  \begin{align*}
   \textsf{v}_n=&\frac{1}{n-1} \sum_{i=1}^n (\rvx_i-\mu+\mu-\textsf{m}_n)^2\\
   =& \frac{1}{n-1} \sum_{i=1}^n (\rvx_i-\mu+\mu-\textsf{m}_n)^2\\
   =& \frac{1}{n-1}\left(\sum_{i=1}^n (x_i-\mu)^2-n(\textsf{m}_n-\mu)^2\right)\Rightarrow
  \end{align*}
  \begin{align*}
   \E[\textsf{v}_n]=&\frac{1}{n-1}\left(\sum_{i=1}^n \E[(x_i-\mu)^2]-n\E[(\textsf{m}_n-\mu)^2]\right)\\
   =& \frac{1}{n-1}(n\sigma^2-\sigma^2)\\
   =& \sigma^2
  \end{align*}  
  因此 $\textsf{v}_n$ 是 $\sigma^2$ 的无偏估计量。
  
  \item 
  \begin{enumerate}[label=(\alph*)]
  \item 
  $\hat{x}_{\textrm{BLS}}(\rvy)=\rvy^{\frac{1}{3}}$
    
  \item
  由LLS估计的公式:
  \[
   \hat{\rvx}_{\textrm{LLS}}(\rvy)=\mu_{\rvx} + \Lambda_{\rvx\rvy}\Lambda^{-1}_{\rvy}(\rvy-\mu_{\rvy})
  \]
  代入   $\mu_{\rvx}=g_1, \mu_{\rvy}=g_3,\Lambda_{\rvy}=g_6-g_3^2,\Lambda_{\rvx\rvy}=g_4-g_1g_3$
  所以
  \[
  \hat{\rvx}_{\textrm{LLS}}(\rvy) = g_1+\frac{g_4-g_1g_3}{g_6-g_3^2}(\rvy-g_3)
  \]
  因为$x$的密度函数关于$0$对称,$g_1=g_3=0\Rightarrow \hat{\rvx}_{\textrm{LLS}}(\rvy) = \frac{g_4}{g_6}\rvy$
  \item 
%  设$\urvy=[\rvy,\rvz_1,\rvz_2]^\TT$  
%  沿用LLS估计公式的向量形式:
%  \[
%   \hat{\rvx}_{\textrm{LLS}}(\urvy)=\mu_{\rvx} + \bm{\Lambda}_{\rvx\urvy}\bm{\Lambda}^{-1}_{\urvy}(\urvy-\mu_{\urvy})
%  \]
%  可以求出$\mu_{\urvy}=[g_3,g_1,g_1]^\TT$,
%  \[
%  \bm{\Lambda}_{\urvy}=\begin{bmatrix}
%  g_6-g_3^2 & g_4-g_3g_1 & g_4-g_3g_1 \\
%  g_4-g_3g_1 & g_2-g_1^2+1 & g_2-g_1^2-2 \\
%  g_4-g_3g_1 & g_2-g_1^2-2 & g_2-g_1^2+4
%  \end{bmatrix}
%  \]
%  $\bm{\Lambda}_{\rvx\urvy}=[g_4-g_1g_3,g_2,g_2]$,
%  所以有:
%    \[
%   \hat{\rvx}_{\textrm{LLS}}(\urvy)=g_1+ [g_4-g_1g_3,g_2,g_2]
%   \begin{bmatrix}
%  g_6-g_3^2 & g_4-g_3g_1 & g_4-g_3g_1 \\
%  g_4-g_3g_1 & g_2-g_1^2+1 & g_2-g_1^2-2 \\
%  g_4-g_3g_1 & g_2-g_1^2-2 & g_2-g_1^2+4
%  \end{bmatrix}^{-1}\begin{bmatrix}
%  \rvy-g_3 \\
%  \rvz_1-g_1\\
%  \rvz_2-g_1
%  \end{bmatrix}
%  \]
  \[
   \hat{\rvx}_{\textrm{LLS}}(\urvy)=\frac{2}{3} \rvz_1 + \frac{1}{3} \rvz_2
  \]
   并且$\MSE[\hat{\rvx}_{\textrm{LLS}}]=0$
  \item $\rvx$的后验概率是各以$\frac{1}{2}$概率等于$\pm \sqrt{\mathsf{v}}$,
  所以 $\hat{x}_{\textrm{BLS}}(\mathsf{v})=0$
  \end{enumerate} 
  \item
  \begin{enumerate}[label=(\alph*)]
  \item 反设存在这样的估计量 $\hat{x}(\rvy)$,那么$\forall x>0$,下式成立:
  \[
  \int_{\mathbb{R}}\hat{x}(y)p_{\rvy}(y;x)dy=x
  \]
  代入题目中已知的概率分布得:
  \[
  \int_0^{\frac{1}{x}} \hat{x}(y)dy = 1 
  \]
  由于$x$是任意的,可以推出$\hat{x}(y)$在正半轴任意区间积分为0,取正半轴的左端点趋近0得矛盾(积分为零不为1)。因此对于给定的分布
  不存在关于$x$的无偏估计量。

  \item
    存在 $ \hat{x}(\rvy)=2\rvy$, 
    
    \[
    \E[\hat{x}(\rvy)] = \frac{1}{x}\int_0^{x} 2 y d y = x
    \]
    因此 $\hat{x}(\rvy)$ 是无偏的。
    又 $\Var[\hat{x}(\rvy)]=x^2$,恰好达到CRB下界 $\frac{1}{J_{\rvy}(x)}$
    其中$J_{\rvy}(x)$按定义式计算:
    \[
      J_{\rvy}(x) = \E\left[\left(\frac{\partial}{\partial x} \ln p_{\rvy}(y;x) \right)^2\right]
    \]
    
    \end{enumerate}
  \item $x[0],x[1],\dots,x[N-1]$的似然函数为:
  
      \[
      \ln p_{\urvx}(\urvx;a)=\frac{-1}{2\sigma^2}\sum_{i=0}^{N-1} (x[i]-ar^i)^2+c
      \]
  
      \begin{align*}
      J_{\urvx}(a)=&-\E[\frac{\partial^2}{\partial a^2}\ln p_{\urvx}(\urvx;a)]\\
      = & \frac{1}{\sigma^2}\sum_{i=0}^{N-1} r^{2i}
      \end{align*}
  \[
  \hat{a}(\urvx)=\frac{\sum_{i=0}^{N-1} x[i] r^i}{\sum_{i=0}^{N-1} r^{2i}}
  \]
  可以验证
  \[
  \Var[\hat{a}(\urvx)]=\frac{1}{J_{\urvx}(a)}
  \]
  因此构造的$\hat{a}(\urvx)$ 是有效估计量。
  $c$是与$a$无关的常数。
  \item Thanks to 陆石, who gives me this template.
  

  \end{enumerate}

\end{document}
\begin{equation}
\end{equation}

%%% Local Variables:
%%% mode: late\rvx
%%% TeX-master: t
%%% End:
