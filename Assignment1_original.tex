\documentclass{article}
\usepackage{xeCJK}
\usepackage{amsmath, amsthm, amssymb}
\usepackage{hyperref}
\setCJKmainfont[AutoFakeBold]{SimSun}
\usepackage{bm}
\DeclareMathOperator{\E}{\mathbb{E}}
\DeclareMathOperator{\Var}{Var}
\DeclareMathOperator{\MSE}{MSE}
\DeclareMathOperator{\Cov}{Cov}

\DeclareMathOperator*{\argmin}{\arg\,\min}

\begin{document}
\title{第一次作业}
\author{赵丰,2017310711}
\maketitle
\textbf{P1.1(a)i.}
\begin{equation}
\E[x|y]=\sum_{i} x_i\Pr(x=x_i|y)
\end{equation}
\begin{align*}
\E[\E[x|yz]|y] = & \E((\sum_{i} x_i \Pr(x=x_i| yz))|y)\\
=&\sum_j (\sum_{i} x_i \Pr(x=x_i| y,z=z_j))\Pr(z=z_j) \\
=&\sum_i x_i(\sum_{j} \Pr(x=x_i| y,z=z_j))\Pr(z=z_j) \\
=&\sum_i x_i(\sum_{j} \Pr(x=x_i,z=z_j| y))\\
=&\sum_i x_i \Pr(x=x_i| y) \\
\end{align*}
所以
\begin{equation}
\E[x|y]=\E[\E[x|yz]|y]
\end{equation}

\textbf{P1.1(a)ii.}
\begin{equation}
g(y)\E[x|y]=\sum_{i} g(y)x_i\Pr(x=x_i|y)=\E[xg(y)|y]
\end{equation}

\textbf{P1.1(a)iii.}
\begin{align*}
\E[x\E[x|y]]=&\E\left[x\sum_{i} x_i\Pr(x=x_i|y)\right]\\
=&\sum_{j,k}(x_j\sum_{i} x_i\Pr(x=x_i|y=y_k))\Pr(x=x_j,y=y_k)\\
=&\sum_{i,j,k} x_i x_j \Pr(x=x_i|y=y_k)\Pr(x=x_j,y=y_k)
\end{align*}
\begin{align*}
\E[(\E[x|y])^2]=&\E\left[(\sum_{i} x_i\Pr(x=x_i|y))^2\right]\\
=&\sum_{k}(\sum_{i} x_i\Pr(x=x_i|y=y_k))^2 \Pr(y=y_k)\\
=&\sum_{i,j,k} x_i x_j\Pr(x=x_i|y=y_k)\Pr(x=x_j|y=y_k)\Pr(y=y_k)\\
=&\sum_{i,j,k} x_i x_j\Pr(x=x_i|y=y_k)\Pr(x=x_j,y=y_k)
\end{align*}
所以
\begin{equation}
\E[x\E[x|y]]=\E[(\E[x|y])^2]
\end{equation}

\textbf{P1.1(a)iv.}
对离散型随机变量证明全方差公式(Law of Total Variance)
用到全期望公式(Law of Total Expectation):
\begin{equation}
\E[\E[x|y]]=\E[x]
\end{equation}

\begin{equation}
\Var(x)=\sum_i (x_i-\E(x))^2\Pr(x=x_i)
\end{equation}

\begin{align*}
\Var(x)=& \E[x^2]-(\E[x])^2\\
=& \E[\E[x^2|y]]-(\E[\E[x|y]])^2\\
=& \E[\Var[x|y]+(\E[x|y])^2]-(\E[\E[x|y]])^2\\
=& \E[\Var[x|y]]+\Var[\E[x|y]]\\
\end{align*}

\textbf{P1.1(b)i.}

\begin{equation}
\E[(y-\alpha)^2]=\alpha^2-2\E[y]\alpha+\E[y]^2
\end{equation}
由二次函数极值的性质,当$\alpha=\E[y]$时 $\E[(y-\alpha)^2]$最小,
此时$\MSE(\hat{y}=\E[y])=\Var(y)$

\textbf{P1.1(b)ii.}
\begin{equation}\label{eq:P11bii}
\E[(y-\hat{y})^2]=\sum_{i}\left(\sum_j (y_j-\hat{y}(x_i))^2\Pr(y=y_j|x=x_i)\right)\Pr(x=x_i)
\end{equation}
故只需对每一个$\Pr(x=x_i)$,极小化
\begin{equation}
%\MSE(\hat{y})(x_i)=\argmin_{f:\mathcal{X}\rightarrow \mathbb{R}}
\left(\sum_j (y_j-f(x_i))^2\Pr(y=y_j|x=x_i)\right)
\end{equation}
目标函数可化为如下形式:
\begin{equation}
\mathcal{F}(f)=f(x_i)^2-2\E[y|x=x_i]f(x_i)+c,f:\mathcal{X}\rightarrow \mathbb{R}
\end{equation}

由第一问结论:
\begin{equation}
%\MSE(\hat{y})(x_i)
\argmin_{f:\mathcal{X}\rightarrow \mathbb{R}}\mathcal{F}(f)=\E[y|x=x_i]
\end{equation}
从而有
\begin{equation}
\argmin_{f:\mathcal{X}\rightarrow \mathbb{R}}\E[(y-f(x))^2]=\E[y|x]
\end{equation}
由\eqref{eq:P11bii}式得估计的MSE为
\begin{equation}
\MSE(\E[y|x])=\sum_{i}\Var(y|x=x_i)\Pr(x=x_i)=\E[\Var(y|x)]
\end{equation}

\textbf{P1.1(b)iii.}
由全方差公式
\begin{equation}
\E[\Var(y|x)]=\Var(y)-\Var(\E[y|x])
\end{equation}
若$x$与$y$相互独立,$\E[y|x]=\E[y]$为常数,因此$\Var(\E[y|x])=0$,所以$\E[\Var(y|x)]=\Var(y)$,即
\begin{equation}
\MSE(\E[y])=\MSE(\E[y|x])
\end{equation}
若上式成立,可知$\E[y|x]=\E[y]$为常数,即
\begin{equation}
\forall i,\sum_{j}y_j\Pr(y=y_j|x=x_i)=\sum_{j}y_j\Pr(y=y_j)
\end{equation}
上式两边同时乘以$\E[f(x)]$可得
\begin{equation}
\sum_{i,j}f(x_i)y_j\Pr(x=x_i,y=y_j)=\left(\sum_i f(x_i)\Pr(x=x_i)\right)\left(\sum_j y_j\Pr(y=y_j)\right)
\end{equation}
即$\E[f(x)y]=\E[f(x)]\E[y]$
所以$\forall f,\rho(f(x),y)=0$

\textbf{1.2(a)}

\begin{equation}
\E[(y-ax-b)^2]==a^2\E[x^2]+\E[y^2]+b^2-2a\E[xy]-2b\E[y]+2ab\E[x]
\end{equation}
对$a,b$求偏导得:
\begin{equation}
\begin{cases}
\E[x^2]a+\E[x]b=&\E[xy]\\
\E[x]a+b=&\E[y]\\
\end{cases}
\end{equation}
从而解出,并考虑到$\Var(x)=\Var(y)=\sqrt{\Var(x)\Var(y)}$:
\begin{equation}
a=\frac{\E[xy]-\E[x]\E[y]}{\Var(x)}=\rho(x,y)
\end{equation}

\textbf{1.2(b)}
\begin{equation}
x \perp y \Longleftrightarrow \Pr(x=x_i,y=y_j)=\Pr(x=x_i)\Pr(y=y_j)
\end{equation}
所以当$x$与$y$独立时
\begin{align*}
\E[f(x)g(y)]=&\sum_{i,j}f(x_i)g(y_j)\Pr(x=x_i,y=y_j)\\
=&\sum_{i,j}f(x_i)g(y_j)\Pr(x=x_i)\Pr(y=y_j)\\
=&\left(\sum_{i}f(x_i)\Pr(x=x_i)\right)\left(\sum_{i}g(y_j)\Pr(y=x_j)\right)\\
=&\E[f(x)]\E[g(y)]
\end{align*}
即$\forall f,g,\rho(f(x),g(y))=0$

另一方面,若$\forall f,g,\rho(f(x),g(y))=0$。取$f=1_{x=x_i},g=1_{y=y_j}$,则由
$\E[f(x)g(y)]=\E[f(x)]\E[g(y)]$可以推出
\begin{equation}
\Pr(x=x_i,y=y_j)=\Pr(x=x_i)\Pr(y=y_j)
\end{equation}
所以$x$与$y$相互独立。
\begin{equation}
\end{equation}
\begin{equation}
\end{equation}
\begin{equation}
\end{equation}
\begin{equation}
\end{equation}

\end{document}
