% Homework template for Inference and Information
% UPDATE: September 26, 2017 by Xiangxiang
\documentclass[a4paper]{article}
\usepackage{amsmath, amssymb, amsthm}
% amsmath: equation*, amssymb: mathbb, amsthm: proof
\usepackage{moreenum}
\usepackage{mathtools}
\usepackage{url}
\usepackage{enumitem}
\usepackage{graphicx}
\usepackage{subcaption}
\usepackage{booktabs} % toprule
\usepackage[mathcal]{eucal}
%% Definitions for Inference and Information
%% UPDATE: 22/03/2018  by zhaofeng-shu33
%% This package does not require other packages
\ProvidesPackage{dtx-style}
% semester
\newcommand{\theterm}[1]{#1}
\newcommand{\thecourseinstitute}[1]{#1}
\newcommand{\thecoursename}[1]{\textsc{#1}}

\newcommand{\courseheader}[3]{
\vspace*{-1in}
\begin{center}
\thecourseinstitute{#1}\\
\thecoursename{#3} \\
\theterm{#2}
\vspace*{0.1in}
\hrule
\end{center}
}
\def\v#1{\underline{#1}}
\newcommand{\rvx}{\mathsf{x}}    % x, r.v.
\newcommand{\rvy}{\mathsf{y}}    % y, r.v.
\newcommand{\rvz}{\mathsf{z}}    % z, r.v.
\newcommand{\rvH}{\mathsf{H}} 
\newcommand{\uy}{\underline{y}}
\newcommand{\urvx}{\underline{\mathsf{x}}}    % x, r.v. vec
\newcommand{\urvy}{\underline{\mathsf{y}}}    % y, r.v. vec
\newcommand{\urvz}{\underline{\mathsf{z}}}    % z, r.v. vec
\newcommand{\urvw}{\underline{\mathsf{w}}} 
\newcommand{\defas}{\triangleq} %\coloneqq
\newcommand{\reals}{\mathbb{R}}
\newcommand{\TT}{\mathrm{T}}    % transpose
% \newcommand{\E}[1]{\mathbb{E}\left[{#1}\right]}
% \newcommand{\Prob}[1]{\mathbb{P}\left({#1}\right)}
\DeclareMathOperator*{\argmax}{arg\,max}
\DeclareMathOperator*{\argmin}{arg\,min}
\DeclareMathOperator*{\argsup}{arg\,sup}
\DeclareMathOperator*{\arginf}{arg\,inf}
\DeclareMathOperator{\diag}{diag}
\DeclareMathOperator{\Var}{Var}
\DeclareMathOperator{\Cov}{Cov}
\DeclareMathOperator{\MSE}{MSE}
\DeclareMathOperator{\In}{\mathbb{I}}
\DeclareMathOperator{\E}{\mathbb{E}}
\DeclareMathOperator{\Prob}{\mathbb{P}}
\newcommand\independent{\protect\mathpalette{\protect\independenT}{\perp}}
\def\independenT#1#2{\mathrel{\rlap{$#1#2$}\mkern2mu{#1#2}}}
\endinput

\begin{document}
\courseheader

\newcounter{hwcnt}
\setcounter{hwcnt}{1} % set to the times of Homework

\begin{center}
  \underline{\bf Homework \thehwcnt} \\
\end{center}
\begin{flushleft}
  赵丰\hfill
  \today
\end{flushleft}
\hrule

\vspace{2em}

\flushleft
\rule{\textwidth}{1pt}
\begin{itemize}
\item {\bf Acknowledgments: \/} 
  This coursework referes to wikipedia: \small{\url{https://en.wikipedia.org}}.

\item {\bf Collaborators: \/}
  I finish this template by myself.
\end{itemize}
\rule{\textwidth}{1pt}

\vspace{2em}

I use \texttt{enumerate} to generate answers for each question:

\begin{enumerate}[label=\thehwcnt.\arabic*.]
  \setlength{\itemsep}{3\parskip}

  \item 
    \begin{enumerate}[label=(\alph*)]
    \item 
    \begin{enumerate}[label=\roman*.]
      \item 
        \begin{equation}
        \E[\rvx|\rvy]=\sum_{i} x_i\Pr(\rvx=x_i|\rvy)
        \end{equation}
        \begin{align*}
        \E[\E[\rvx|\rvy \rvz]|\rvy] = & \E((\sum_{i} x_i \Pr(\rvx=x_i| \rvy \rvz))|\rvy)\\
        =&\sum_j (\sum_{i} x_i \Pr(\rvx=x_i| \rvy,\rvz=z_j))\Pr(\rvz=z_j) \\
        =&\sum_i x_i(\sum_{j} \Pr(\rvx=x_i| \rvy,\rvz=z_j))\Pr(\rvz=z_j) \\
        =&\sum_i x_i(\sum_{j} \Pr(\rvx=x_i,\rvz=z_j| \rvy))\\
        =&\sum_i x_i \Pr(\rvx=x_i| \rvy) \\
        \end{align*}
        所以
        \begin{equation}
        \E[\rvx|\rvy]=\E[\E[\rvx|\rvy \rvz]|\rvy]
        \end{equation}

      \item 
        \begin{equation}
        g(\rvy)\E[\rvx|\rvy]=\sum_{i} g(\rvy)x_i\Pr(\rvx=x_i|\rvy)=\E[xg(\rvy)|\rvy]
        \end{equation}
      
      \item
        \begin{align*}
        \E[\rvx\E[\rvx|\rvy]]=&\E\left[\rvx\sum_{i} x_i\Pr(\rvx=x_i|\rvy)\right]\\
        =&\sum_{j,k}(x_j\sum_{i} x_i\Pr(\rvx=x_i|\rvy=y_k))\Pr(\rvx=x_j,\rvy=y_k)\\
        =&\sum_{i,j,k} x_i x_j \Pr(\rvx=x_i|\rvy=y_k)\Pr(\rvx=x_j,\rvy=y_k)
        \end{align*}
        \begin{align*}
        \E[(\E[\rvx|\rvy])^2]=&\E\left[(\sum_{i} x_i\Pr(\rvx=x_i|\rvy))^2\right]\\
        =&\sum_{k}(\sum_{i} x_i\Pr(\rvx=x_i|\rvy=y_k))^2 \Pr(\rvy=y_k)\\
        =&\sum_{i,j,k} x_i x_j\Pr(\rvx=x_i|\rvy=y_k)\Pr(\rvx=x_j|\rvy=y_k)\Pr(\rvy=y_k)\\
        =&\sum_{i,j,k} x_i x_j\Pr(\rvx=x_i|\rvy=y_k)\Pr(\rvx=x_j,\rvy=y_k)
        \end{align*}
        所以
        \begin{equation}
        \E[\rvx\E[\rvx|\rvy]]=\E[(\E[\rvx|\rvy])^2]
        \end{equation}
      \item
        对离散型随机变量证明全方差公式(Law of Total Variance)
        用到全期望公式(Law of Total Expectation):
        \begin{equation}
        \E[\E[\rvx|\rvy]]=\E[\rvx]
        \end{equation}

        \begin{equation}
        \Var(\rvx)=\sum_i (x_i-\E(\rvx))^2\Pr(\rvx=x_i)
        \end{equation}

        \begin{align*}
        \Var(\rvx)=& \E[\rvx^2]-(\E[\rvx])^2\\
        =& \E[\E[\rvx^2|\rvy]]-(\E[\E[\rvx|\rvy]])^2\\
        =& \E[\Var[\rvx|\rvy]+(\E[\rvx|\rvy])^2]-(\E[\E[\rvx|\rvy]])^2\\
        =& \E[\Var[\rvx|\rvy]]+\Var[\E[\rvx|\rvy]]\\
        \end{align*}      
    \end{enumerate}
    \item 
      \begin{enumerate}[label=\roman*.]
        \item
            \begin{equation}
            \E[(\rvy-\alpha)^2]=\alpha^2-2\E[\rvy]\alpha+\E[\rvy]^2
            \end{equation}
            由二次函数极值的性质,当$\alpha=\E[\rvy]$时 $\E[(\rvy-\alpha)^2]$最小,
            此时$\MSE(\hat{\rvy}=\E[\rvy])=\Var(\rvy)$
        \item
            \begin{equation}\label{eq:P11bii}
            \E[(\rvy-\hat{\rvy})^2]=\sum_{i}\left(\sum_j (y_j-\hat{\rvy}(x_i))^2\Pr(\rvy=y_j|\rvx=x_i)\right)\Pr(\rvx=x_i)
            \end{equation}
            故只需对每一个$\Pr(\rvx=x_i)$,极小化
            \begin{equation}
            %\MSE(\hat{y})(x_i)=\argmin_{f:\mathcal{X}\rightarrow \mathbb{R}}
            \left(\sum_j (y_j-f(x_i))^2\Pr(\rvy=y_j|\rvx=x_i)\right)
            \end{equation}
            目标函数可化为如下形式:
            \begin{equation}
            \mathcal{F}(f)=f(x_i)^2-2\E[\rvy|\rvx=x_i]f(x_i)+c,f:\mathcal{X}\rightarrow \mathbb{R}
            \end{equation}

            由第一问结论:
            \begin{equation}
            %\MSE(\hat{y})(x_i)
            \argmin_{f:\mathcal{X}\rightarrow \mathbb{R}}\mathcal{F}(f)=\E[\rvy|\rvx=x_i]
            \end{equation}
            从而有
            \begin{equation}
            \argmin_{f:\mathcal{X}\rightarrow \mathbb{R}}\E[(\rvy-f(\rvx))^2]=\E[\rvy|\rvx]
            \end{equation}
            由\eqref{eq:P11bii}式得估计的MSE为
            \begin{equation}
            \MSE(\E[\rvy|\rvx])=\sum_{i}\Var(\rvy|\rvx=x_i)\Pr(\rvx=x_i)=\E[\Var(\rvy|\rvx)]
            \end{equation}
        \item
            由全方差公式
            \begin{equation}
            \E[\Var(\rvy|\rvx)]=\Var(\rvy)-\Var(\E[\rvy|\rvx])
            \end{equation}
            若$\rvx$与$\rvy$相互独立,$\E[\rvy|\rvx]=\E[\rvy]$为常数,因此$\Var(\E[\rvy|\rvx])=0$,所以$\E[\Var(\rvy|\rvx)]=\Var(\rvy)$,即
            \begin{equation}
            \MSE(\E[\rvy])=\MSE(\E[\rvy|\rvx])
            \end{equation}
            若上式成立,可知$\E[\rvy|\rvx]=\E[\rvy]$为常数,即
            \begin{equation}
            \forall i,\sum_{j}y_j\Pr(\rvy=y_j|\rvx=x_i)=\sum_{j}y_j\Pr(\rvy=y_j)
            \end{equation}
            上式两边同时乘以$\E[f(\rvx)]$可得
            \begin{equation}
            \sum_{i,j}f(x_i)y_j\Pr(\rvx=x_i,\rvy=y_j)=\left(\sum_i f(x_i)\Pr(\rvx=x_i)\right)\left(\sum_j y_j\Pr(\rvy=y_j)\right)
            \end{equation}
            即$\E[f(\rvx)\rvy]=\E[f(\rvx)]\E[\rvy]$
            所以$\forall f,\rho(f(\rvx),\rvy)=0$                
      \end{enumerate}      
    \end{enumerate}
\item 
  \begin{enumerate}[label=(\alph*)]
  \item
    \begin{equation}
    \E[(\rvy-a\rvx-b)^2]==a^2\E[\rvx^2]+\E[\rvy^2]+b^2-2a\E[\rvx\rvy]-2b\E[\rvy]+2ab\E[\rvx]
    \end{equation}
    对$a,b$求偏导得:
    \begin{equation}
    \begin{cases}
    \E[\rvx^2]a+\E[\rvx]b=&\E[\rvx\rvy]\\
    \E[\rvx]a+b=&\E[\rvy]\\
    \end{cases}
    \end{equation}
    从而解出,并考虑到$\Var(\rvx)=\Var(\rvy)=\sqrt{\Var(\rvx)\Var(\rvy)}$:
    \begin{equation}
    a=\frac{\E[\rvx\rvy]-\E[\rvx]\E[\rvy]}{\Var(\rvx)}=\rho(\rvx,\rvy)
    \end{equation}
  \item
    \begin{equation}
    \rvx \perp \rvy \Longleftrightarrow \Pr(\rvx=x_i,\rvy=y_j)=\Pr(\rvx=x_i)\Pr(\rvy=y_j)
    \end{equation}
    所以当$\rvx$与$\rvy$独立时
    \begin{align*}
    \E[f(\rvx)g(\rvy)]=&\sum_{i,j}f(x_i)g(\rvy_j)\Pr(\rvx=x_i,\rvy=y_j)\\
    =&\sum_{i,j}f(x_i)g(y_j)\Pr(\rvx=x_i)\Pr(\rvy=y_j)\\
    =&\left(\sum_{i}f(x_i)\Pr(\rvx=x_i)\right)\left(\sum_{i}g(y_j)\Pr(\rvy=x_j)\right)\\
    =&\E[f(\rvx)]\E[g(\rvy)]
    \end{align*}
    即$\forall f,g,\rho(f(\rvx),g(\rvy))=0$

    另一方面,若$\forall f,g,\rho(f(\rvx),g(\rvy))=0$。取$f=1_{\rvx=x_i},g=1_{\rvy=y_j}$,则由
    $\E[f(\rvx)g(\rvy)]=\E[f(\rvx)]\E[g(\rvy)]$可以推出
    \begin{equation}
    \Pr(\rvx=x_i,\rvy=y_j)=\Pr(\rvx=x_i)\Pr(\rvy=y_j)
    \end{equation}
    所以$\rvx$与$\rvy$相互独立。
  \end{enumerate}
  \item Thanks to 陆石, who gives me this template.
  

  \end{enumerate}

\end{document}
%%% Local Variables:
%%% mode: late\rvx
%%% TeX-master: t
%%% End:
